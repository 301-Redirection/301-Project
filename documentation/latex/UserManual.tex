\documentclass{article}
\usepackage[utf8]{inputenc}
\usepackage{graphicx}
\usepackage{hyperref}
\begin{document}

\title{User manual}
\date{2018}
\author{COS 301: REDIRECTION}
\maketitle
\pagebreak
\tableofcontents
\pagebreak

\section{System Overview}
The project is designed to provide a user-friendly and intuitive system that gives users the ability to configure bot scripts for the computer video game called DOTA 2. The bot scripts allow a player to manipulate the behaviour and game playing strategies of the computer controlled bots within the game. The user will configure these bot scripts according to a set of predefined characteristics such as the desire to "push a lane", "defend", "roam" and  defeat "Roshan" just to name a few. This functionality will allow a non-technical user to be able to manipulate the bots behaviour as well as control how the bots act within the game without any knowledge of coding and other technical requirements required for bot scripting. The different options will allow the user to influence the way in which the game is played by the bot team, from a macro level such as the expected length of the game, to a micro level, such as the choices the bots make when purchasing items and upgrading their character’s abilities. The configuration options selected by the user will be converted to LUA scripts (Programming language used to interface with the DOTA 2 bot scripting API) and the user will be able to download their newly generated scripts. The user will then be able to follow the installation guide which will show them how to install the generated script and then how to use their newly installed scripts.
\\\\
The system is in place to allow more players to be able to customise their DOTA 2 game play experiences by providing a simple and easy way of creating custom bot scripts and providing the user with assistance in the installation process of these bot scripts. 

	
\section{System Configuration}
**Describe and depict graphically the equipment, communications, and networks used by
the system. Include the type and configuration of computer input and output devices.**\newpage

\subsubsection*{Deployment Diagram:}
	\addcontentsline{toc}{subsection}{Deployment Diagram}
Our system makes use of the following:\\
\includegraphics[width=\textwidth]{Resources/Diagrams/Deployment_Diagram.png}
\newpage

\section{Installation}
**Detailed description of where to find the software and how to install it.**\\
\begin{flushleft}
Initially the software is not installed since it is a website, so during the phases of designing and editing your bot scripts there is no software required.\\
Once you have designed your bot and downloaded it using the tutorials below, there is a guide on where to put the bot and how to use it.
\end{flushleft}

\section{Getting Started}
** Describe the procedures necessary to access the system, including how to get a licence,
user ID and log on. Describe how the user changes a user ID. Describe the actions a user
must take to change a password.\\
Provide a general walk through of the system from initiation through exit. The logical
arrangement of the information should enable the potential user to understand the se-
quence and flow of the system. Use screen captures to depict examples. Give reference
to where these are explained in more detail the manual.\\
Describe the actions necessary to properly exit the system.** 
\subsection*{Lets get started, follow the 6 easy steps to SIGN UP:}
	\addcontentsline{toc}{subsection}{Sign Up}
\begin{flushleft}
\textbf{\\1. Click the following website link to visit the website: ** Insert url to website here **}\\
\textbf{\\2. Click the login button on the top right navigation bar:}
\includegraphics[width=\textwidth]{Resources/Tutorial-Images/signup-1.png}\\
\textbf{\\3. You will be redirected to our authentication provider where you will select the sign up option:}
\includegraphics[width=\textwidth]{Resources/Tutorial-Images/signup-2.png}
\\You are given the choice to sign up using a pre-existing Google or Facebook account for convenience sake. If you do not have either, you can still sign up by entering in a valid email address and a password that fulfils the following security criteria, passwords must contain: at least 8 characters, lower and upper case letters and numbers (0-9) and special characters.\\
To login using Google or Facebook simply click on the relevant icon. This will redirect you to their login page, where you simply login as you normally would. Once logged in it will redirect you to the website again, and proceed with the sign up process.
\textbf{\\4. After filling your details click sign up}
\includegraphics[width=\textwidth]{Resources/Tutorial-Images/signup-3.png}\\
\textbf{\\5. You will need to authorise our application to access basic information about your profile}
\includegraphics[width=\textwidth]{Resources/Tutorial-Images/signup-4.png}\\
\textbf{\\6. You are now registered for our product, lets create your first Dota 2 bot script}
\includegraphics[width=\textwidth]{Resources/Tutorial-Images/signup-5.png}
You are now registered for out product, you can now create Dota 2 bot scripts with the ease of a click of a button, now you don't need to be an expert to be able to play like one. To create your first bot script follow the steps described in create your first bot script tutorial. 
\end{flushleft}
\subsection*{LOGIN to our product, follow the 5 steps:}
	\addcontentsline{toc}{subsection}{Login}
\begin{flushleft}
\textbf{\\1. Click the following website link to visit the website: ** Insert url to website here **}\\
\textbf{\\2. Click the login button on the top right navigation bar:}
\includegraphics[width=\textwidth]{Resources/Tutorial-Images/signup-1.png}\\
\textbf{\\3. You will be redirected to our authentication provider where you will select your login option:}
\includegraphics[width=\textwidth]{Resources/Tutorial-Images/login-1.png}
\\You are given the choice to login using a pre-existing Google or Facebook account for convenience sake. If you do not have either, you can still login by entering in the email address and a password used to register for our product.\\
\textbf{\\4. After filling your details click login}
\includegraphics[width=\textwidth]{Resources/Tutorial-Images/login-1.png}
\textbf{\\5. If login was successful you will be redirected to the homepage}
\includegraphics[width=\textwidth]{Resources/Tutorial-Images/signup-5.png}
You are logged in for out product, you can now create Dota 2 bot scripts with the ease of a click of a button, now you don't need to be an expert to be able to play like one. To create your first bot script follow the steps described in create your first bot script tutorial. 
\end{flushleft}



\section{Using the System}
** This section forms the bulk of the user manual. It consists of a detailed description of the
system functions. Each function (use case) should be described. Include screen captures
and descriptive narrative to explain when, why and how each function is used.\\
If the system has query or retrieval capabilities, the instructions necessary for recognition,
preparation, and processing of a query applicable to a database should be explained in
detail. If the system can create reports, describe all reports available to the end user.
Include report format and the meaning of each field shown on the report.\\
Use appropriate data examples to explain the full extent in which each function can be
used. **
\subsection*{CREATE a DOTA 2 bot script, follow the 8 steps:}
	\addcontentsline{toc}{subsection}{Creating Scripts}
\begin{flushleft}
\textbf{\\1. You need to be logged in to create bot scripts, if not please follow the login tutorial}\\

\textbf{\\2. From any page click "New bot configuration" on the navigation bar at the top of the page}
\includegraphics[width=\textwidth]{Resources/Tutorial-Images/create-bot-1.png}\\

\textbf{\\3. You will be redirected to the bot configuration page:}
\includegraphics[width=\textwidth]{Resources/Tutorial-Images/create-bot-2-1.png}
\\On the bot script configuration page you can give a bot script a name and description, you are then required to choose whether the script will edit either the Radiant, the Dire or both factions bot behaviour, to edit only on of the 2 factions click on either the top half or the bottom half of the map titled "Edit a particular faction", to edit both factions simply click the map titled "Edit for both factions".
\includegraphics[width=\textwidth]{Resources/Tutorial-Images/create-bot-2-2.png}
Once you are happy with your details and selections we move on to the next step, we do this by clicking the "Team Desires" tab near the top of the page.

\textbf{\\4. Setting the team desires}
\includegraphics[width=\textwidth]{Resources/Tutorial-Images/create-bot-3-1.png}
\\For team desires you are presented with 5 tabs which are:
\begin{itemize}
	\item Team's Push Desire
	\item Team's Defend Desire
	\item Team's Farm Desire
	\item Team's Roshan Desire
	\item Team's Roam Desire
\end{itemize}
\textbf{\\4.1 Setting the Team's Push Desire, Team's Defend Desire and Team's Farm Desire}
\includegraphics[width=\textwidth]{Resources/Tutorial-Images/create-bot-3-2.png}
Start by clicking on one of the three tabs to open the options. Once the tab is open you will see 3 more tabs, namely "Top", "Middle" and "Bottom" these tabs correspond to each lane within the game. To start select which lane you wish to edit, then select the default value, this value represents the starting desire, and will be the desire value for the selected lane at the start of the game.\\
\includegraphics[width=\textwidth]{Resources/Tutorial-Images/create-bot-3-3.png}
Now to edit this default desire value through out the course of the game we create conditions. Conditions consist of a \textbf{trigger} (Something happening within the game such as Time or number of enemy heroes alive), an \textbf{operator} (Which represents what the values relation must be to the trigger, such as Less than or Greater than), a \textbf{value} (This is so we can decide what we want the triggers value to be) and then an \textbf{action} (The action determines how this trigger effects the default desire value, it can either modify the default desire or completely overwrite it).\\
\includegraphics[width=\textwidth]{Resources/Tutorial-Images/create-bot-3-4.png}
We can also create compound conditions, this is achieved by clicking the "+" sign shown in the image above. This adds more conditions that need to be met before the modifications are made, such as wanting game time to be past 10 minutes and 3 enemy heroes to be dead.\\
\textbf{\\4.1 Setting the Team's Roshan Desire and Team's Roam Desire}
\includegraphics[width=\textwidth]{Resources/Tutorial-Images/create-bot-3-5.png}
These last 2 tabs are very similar to the first 3 tabs described above, but they just do not have the lane specific aspects since Roaming and Roshan are not lane specific actions.

\textbf{\\5. Hero selection}
\includegraphics[width=\textwidth]{Resources/Tutorial-Images/create-bot-4-1.png}
The hero selection screen allows you to select which heroes you want the bots to select and play as.\\
\includegraphics[width=\textwidth]{Resources/Tutorial-Images/create-bot-4-2.png}
To select the heroes that you want you can either simply \textbf{double click} on the hero, or \textbf{drag} the hero to the "\textbf{Selected}" bar at the bottom of the page. The bots in the game will randomly choose a hero from any of the heroes that you added to the pool during this selection process.\\

\textbf{\\6. Hero specific ability selection}
\includegraphics[width=\textwidth]{Resources/Tutorial-Images/create-bot-5-1.png}
For the next step we select hero-specific ability progress for all the heroes selected in the previous step. To start select which hero's abilities you wish to edit, That hero's abilities will appear in the box below (I will be editing Luna's abilities in the example)\\
\includegraphics[width=\textwidth]{Resources/Tutorial-Images/create-bot-5-2.png}
For this section i will be demonstrating using Luna's abilities\\
Above the "Generate button you can see each of Luna's abilities, You can move and reorder these abilities using the arrows above them, the order in which you put them will determine their priority when you click the generate button.
\includegraphics[width=\textwidth]{Resources/Tutorial-Images/create-bot-5-3.png}
In the image above i prioritised Luna's "Lunar Blessing" ability, followed by her "Lunecent Beam" ability, this caused those 2 abilities to be prioritised when the \textbf{Generate} button was clicked, so hence they will be the first abilities to be levelled up in the game. 
\includegraphics[width=\textwidth]{Resources/Tutorial-Images/create-bot-5-4.png}
Another way to choose the ability progression is to manually select it yourself, instead of clicking on the Generate button we can manually click no the little squares to order the abilities.

\textbf{7. Hero specific Item selection}
\includegraphics[width=\textwidth]{Resources/Tutorial-Images/create-bot-6-1.png}
For the final step in editing the scripts we need to tell the bots what items they must buy. We do this by clicking the \textbf{Items} tab, in this section we choose the hero we want to buy items for. For this example i will be editing Luna again
\includegraphics[width=\textwidth]{Resources/Tutorial-Images/create-bot-6-2.png}
To select which items to buy you can again either \textbf{double click} the icon or \textbf{drag} the item into the "Selected Items" box, The item will then appear in the box, along with its total cost and any items that are required to build the item, if any are required. The total cost also appears at the top of the box.

\textbf{\\8. Save, generate and download your first bot script}
\includegraphics[width=\textwidth]{Resources/Tutorial-Images/create-bot-7.png}
Now that we have finished creating our bots we can finally download and play with the bots. To do this:\\
\textbf{8.1} Click the save button in case we wish to modify the script at a later point\\
\textbf{8.2} Click the generate button to create the LUA bot script file.\\
\textbf{8.3} A download pop up should appear\\
\textbf{8.4} Click the save button to download the bot script file
\textbf{\\You have successfully created and downloaded your first DOTA 2 bot script}
To install and use your bot script please follow the bot script installation guide below. 
\end{flushleft}

\subsection*{Now to install the bot follow these steps:}
	\addcontentsline{toc}{subsection}{Installing Scripts}
\begin{flushleft}
\textbf{1. If you have not already completed the previous tutorial about creating your scripts and downloading them please do so now}

\textbf{2. Start by navigating to your steam directory}
\begin{itemize}
	\item Windows: "C:$\backslash$Program Files (x86)$\backslash$Steam"
	\item Mac: "Users/\textless USERNAME\textgreater/Library/Application Support/Steam"
\end{itemize}
\includegraphics[width=0.8\textwidth]{Resources/Tutorial-Images/install-1.png}

\textbf{3. Once there, if there is no "bots" folder create one (case sensitive)}
\includegraphics[width=0.8\textwidth]{Resources/Tutorial-Images/install-2.png}

\textbf{4. Copy the zipped file from before into the bots folder}
\includegraphics[width=0.8\textwidth]{Resources/Tutorial-Images/install-3.png}

\textbf{5. Extract the zipped file into this directory using software such as \href{https://www.7-zip.org/}{7Zip} or \href{https://www.win-rar.com/}{WinRar}}
\includegraphics[width=0.8\textwidth]{Resources/Tutorial-Images/install-4.png}

\textbf{6. Once complete your directory should look similar to this}
\includegraphics[width=0.8\textwidth]{Resources/Tutorial-Images/install-5.png}

\end{flushleft}


\section{Troubleshooting}
** Describe all recovery and error correction procedures, including error conditions that may
be generated and corrective actions that may need to be taken.\\
If there are special actions the user must take to insure that data is properly saved or
that some other function executes properly, describe those actions here. Include screen
captures and descriptive narratives, if applicable. For example, Show screen shots of error
messages produced by your system and for each explain what may have caused the error
and explain how the user can avoid the error. **


\end{document}
